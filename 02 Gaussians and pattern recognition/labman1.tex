\documentclass[twoside,a4paper,titlepage]{article}
\usepackage{array,amsmath,amssymb,rotating,epsfig,fancyheadings}


\newcommand{\mat}[1]{{\tt >> #1} \\}
\newcommand{\com}[1]{{\tt #1}}
\newcommand{\expl}[1]{%
\begin{turn}{180}%
\parbox{\textwidth}{\em #1}%
\end{turn}%
}
\newcommand{\tit}[1]{{\noindent \bf #1 \\}}
\newcommand{\tab}{\hspace{1em}}

\setlength{\hoffset}{-1in}
\setlength{\voffset}{-1in}

\setlength{\topskip}{0cm}
\setlength{\headheight}{0cm}
\setlength{\headsep}{0cm}

\setlength{\textwidth}{16cm}
\setlength{\evensidemargin}{2.5cm}
\setlength{\oddsidemargin}{2.5cm}

\setlength{\textheight}{24.7cm}
\setlength{\topmargin}{1.5cm}
\setlength{\headheight}{0.5cm}
\setlength{\headsep}{0.5cm}

\pagestyle{fancyplain}

\begin{document}

\begin{titlepage}
\setcounter{page}{-1}

\centerline{Ecole Polytechnique F\'ed\'erale de Lausanne}

\vspace{5cm}

\begin{center}
\huge
Session 1/2\,: \\
Introduction to Gaussian Statistics \\
and Statistical Pattern Recognition
\end{center}

\vfill

\centerline{\includegraphics[width=7cm]{cover.eps.gz}}

\vfill

\noindent
\begin{tabular}{lll}
Course\,: & Speech processing and speech recognition & \\
& & \\
Teacher\,: & Prof. Herv\'e Bourlard \hspace{1cm} & {\tt bourlard@idiap.ch} \\
& & \\
Assistants\,: & Hemant Misra & {\tt misra@idiap.ch}\\
 & Mathew Magimai-Doss & {\tt mathew@idiap.ch} 
\end{tabular}

\end{titlepage}

\thispagestyle{empty}
%%%%%%%%%
%%%%%%%%%
\section*{Guidelines}
%%%%%%%%%
%%%%%%%%%
The following lab manual is structured as follows\,:
\begin{itemize}
\item each section corresponds to a theme
\item each subsection corresponds to a separate experiment.
\end{itemize}
The subsections begin with useful formulas and definitions that will be put
in practice during the experiments. These are followed by the description
of the experiment and by an example of how to realize it in {\sc Matlab}.

If you follow the examples literally, you will be able to progress into the
lab session without worrying about the experimental implementation
details. If you have ideas for better {\sc Matlab} implementations, you are
welcome to put them in practice provided you don't loose too much time\,:
remember that a lab session is no more than 3 hours long.

The subsections also contain questions that you should think about.
Corresponding answers are given right after, in case of problem. You can
read them right after the question, {\em but}\,: the purpose of this lab is
to make you

\medskip
\centerline{\LARGE \bf Think !}
\medskip

If you get lost with some of the questions or some of the explanations, DO
ASK the assistants or the teacher for help\,: they are here to make the course
understood. There is no such thing as a stupid question, and the only
obstacle to knowledge is laziness.

\bigskip
Have a nice lab;

\hfill Teacher \& Assistants \hspace{2cm}


\vfill
%%%%%%%%%
%%%%%%%%%
\section*{Before you begin...}
%%%%%%%%%
%%%%%%%%%
If this lab manual has been handed to you as a hardcopy\,:
\begin{enumerate}
\item get the lab package from \\
	\hspace{2cm}{\tt ftp.idiap.ch/pub/sacha/labs/Session1.tgz}
\item un-archive the package\,: \\
	{\tt \% gunzip Session1.tgz \\
	\% tar xvf Session1.tar}
\item change directory\,: \\
	{\tt \% cd session1}
\item start {\sc Matlab}\,: \\
	{\tt \% matlab }
\end{enumerate}
Then go on with the experiments...


\vspace{1cm}

{\scriptsize
\noindent
This document was created by\,: Sacha Krstulovi\'c ({\tt sacha@idiap.ch}).

\noindent
This document is currently maintained by\,: Sacha Krstulovi\'c ({\tt sacha@idiap.ch}). Last modification on \today.

\noindent
This document is part of the package {\tt Session4.tgz} available by ftp as\,: {\tt ftp.idiap.ch/pub/sacha/labs/Session1.tgz} .
}

\clearpage

\tableofcontents

\bigskip
%\clearpage
%%%%%%%%%
%%%%%%%%%
\section{Gaussian statistics}
%%%%%%%%%
%%%%%%%%%

%%%%%%%%%
\subsection{Samples from a Gaussian density}
\label{samples}
%%%%%%%%%

\subsubsection*{Useful formulas and definitions\,:}
\begin{itemize}
\item[-] The {\em Gaussian probability density function} (Gaussian pdf) for
the $d$-dimensional random variable $x \circlearrowleft {\cal
N}(\mu,\Sigma)$ (i.e. variable $x$ following the Gaussian, or Normal,
probability law) is given by\,:
\[
g_{(\mu,\Sigma)}(x) = \frac{1}{\sqrt{2\pi}^d \sqrt{\det\left(\Sigma\right)}}
\, e^{-\frac{1}{2} (x-\mu)^T \Sigma^{-1} (x-\mu)}
\]
where $\mu$ is the mean vector and $\Sigma$ is the variance-covariance
matrix. $\mu$ and $\Sigma$ are the {\em parameters} of the Gaussian
distribution. {\em Speech features} (also referred as {\em acoustic
vectors}) are examples of $d$-dimensional variables, and it is usually
assumed that they follow a Gaussian distribution.
%
\item[-] If $x \circlearrowleft {\cal N}(0,I)$ ($x$ follows a normal law
with zero mean and unit variance; $I$ denotes the identity matrix), and if
$y = \sqrt{\Sigma} \, x + \mu$, then $y \circlearrowleft {\cal
N}(\mu,\Sigma)$.
%
\item[-] $\sqrt{\Sigma}$ defines the {\em standard deviation} of the random
variable $x$ ({\em \'ecart-type} in French). Beware\,: this square root is
meant in the {\em matrix sense}.
\end{itemize}

\subsubsection*{Experiment\,:}
Generate a sample $X$ of $N$ points, i.e. $X=\{x_1, x_2,\cdots,x_N\}$, with
$N=10000$, coming from a 2-dimensional Gaussian process that has mean\,:
\[
\mu = \left( \begin{array}{c} 730 \\ 1090 \end{array} \right)
\]
and variance\,:
	\begin{enumerate}
	%%%%%
	\item 8000 for both dimensions ({\em spherical process}) (sample $X_1$)\,:
		\[
		\Sigma_1 = \left[ \begin{array}{cc}
						8000 & 0 \\
						0    & 8000
						\end{array} \right]
		\]
	%%%%%
	\item expressed as a {\em diagonal} covariance matrix (sample $X_2$)\,:
		\[
		\Sigma_2 = \left[ \begin{array}{cc}
						8000 & 0 \\
						0    & 18500
						\end{array} \right]
		\]
	%%%%%
	\item expressed as a {\em full} covariance matrix (sample $X_3$)\,:
		\[
		\Sigma_3 = \left[ \begin{array}{cc}
						8000 & 8400 \\
						8400 & 18500
						\end{array} \right]
		\]
	%%%%%
	\end{enumerate}
%
Use the function \com{gausview} (\com{>> help gausview}) to plot the
results as clouds of points in the 2-dimensional plane, and to view the
corresponding 2-dimensional probability density functions (pdfs) in 2D and
3D.

\subsubsection*{Example\,:}
\mat{N = 10000;}
\mat{mu = [730 1090]; sigma\_1 = [8000 0; 0 8000];}
\mat{X1 = randn(N,2) * sqrtm(sigma\_1) + repmat(mu,N,1);}
\mat{gausview(X1,mu,sigma\_1,'Sample X1');}
%
Repeat the three previous steps for the two other Gaussians. Use the radio
buttons to switch the plots on/off. Use the ``view'' buttons to switch
between 2D and 3D. Use the mouse to rotate the plot.


{\bf \noindent Note\,:} if you don't know what one of the cited {\sc Matlab}
command does, use \\
\mat{help {\it command}}
to get some help. If the help doesn't make it clearer, ask the assistants.

\subsubsection*{Question\,:}
By simple inspection of 2D views of the data and of the corresponding pdf
contours, how can you tell which sample corresponds to a spherical process,
which sample corresponds to a pdf with a diagonal covariance, and which to
a pdf with a full covariance~?

\subsubsection*{Answer\,:}
\expl{The cloud of points and the pdf contours corresponding to $\Sigma_1$
are circular, because in this case the first and the second dimension of
the vectors are independent. As a matter of fact, they have a null
covariance\,:
\[ {\cal E}\left[(x_{d1}-\mu_{d1})^T (x_{d2}-\mu_{d2})\right] = 0 \]
They are {\em orthogonal} in the statistical sense, which transposes to a
geometric sense (the expectation is a scalar product of random variables; a
null scalar product means orthogonality). Intuitively, you can also
consider that a null covariance means no sharing of information between the
two dimensions\,: they can evolve independently in a Gaussian way along
their respective axes. Besides, the variance is the same in both
dimensions, which indicates an equivalent spread of the data along both
axes. Hence the circular blob of data points and the name ``spherical
process''.

\medskip

\tab The cloud of points and the pdf contours corresponding to $\Sigma_2$
are elliptic, with their axes parallel to the abscissa and ordinate
axes. This is because the first and the second dimension are still
independent (their covariance is null again), but this time the variance is
different along both dimensions.

\medskip

\tab For $\Sigma_3$, the covariance of both dimensions is not null, so the
principal axes of the ellipses are not aligned with the abscissa and
ordinate axes.}

\pagebreak
%%%%%%%%%
\subsection{Gaussian modeling\,: mean and variance of a sample}
%%%%%%%%%

\subsubsection*{Useful formulas and definitions\,:}
\begin{itemize}
\item[-] Mean estimator\,: $\hat{\mu} = \frac{1}{N} \sum_{i=1}^{N} x_i$
\item[-] Unbiased covariance estimator\,: $ \hat{\Sigma} = \frac{1}{N-1} \;
\sum_{i=1}^{N} (x_i-\mu)^T (x_i-\mu) $
\end{itemize}

\subsubsection*{Experiment\,:}
Take the set $X_3$ of 10000 points generated from ${\cal
N}(\mu,\Sigma_3)$. Compute an estimate $\hat{\mu}$ of its mean and an
estimate $\hat{\Sigma}$ of its variance\,:
\begin{enumerate}
\item with all the available points \hspace{1cm}$\hat{\mu}_{(10000)}
=$\hspace{3.5cm}$\hat{\Sigma}_{(10000)} =$
\vspace{0.8cm}
\item with only 1000 points \hspace{2cm}$\hat{\mu}_{(1000)}
=$\hspace{3.7cm}$\hat{\Sigma}_{(1000)} =$
\vspace{0.8cm}
\item with only 100 points \hspace{2.1cm}$\hat{\mu}_{(100)}
=$\hspace{3.9cm}$\hat{\Sigma}_{(100)} =$
\vspace{0.8cm}
\end{enumerate}
Compare the estimated value $\hat{\mu}$ with the original value of $\mu$ by
measuring the Euclidean distance that separates them. Compare the estimated
value $\hat{\Sigma}$ with the original value of $\Sigma_3$ by measuring the
matrix 2-norm of their difference ($\parallel A-B \parallel_2$ constitutes
a measure of similarity of two matrices $A$ and $B$; use {\sc Matlab}'s
\com{norm} command).

\subsubsection*{Example\,:}
In the case of 1000 points (case 2.)\,: \\
\mat{X = X3(1:1000,:);}
\mat{N = size(X,1)}
\com{>> mu\_1000 = sum(X)/N} \,\,{\it -or-}\,\, \mat{mu\_1000 = mean(X)}
\mat{sigma\_1000 = (X - repmat(mu\_1000,N,1))' * (X - repmat(mu\_1000,N,1)) / (N-1)}
\,\,{\it -or-}\,\, \mat{sigma\_1000 = cov(X)}

\noindent
\mat{\% Comparison of the values:}
\mat{e\_mu =  sqrt( (mu\_1000 - mu) * (mu\_1000 - mu)' )}
\mat{\% (This is the Euclidean distance between mu\_1000 and mu)}
\mat{e\_sigma = norm( sigma\_1000 - sigma\_3 )}
\com{>> \% (This is the 2-norm of the difference between sigma\_1000 and sigma\_3)}

\subsubsection*{Question\,:}
When comparing the estimated values $\hat{\mu}$ and $\hat{\Sigma}$ with the
original values of $\mu$ and $\Sigma_3$ (using the Euclidean distance and
the matrix 2-norm), what can you observe~?

\subsubsection*{Answer\,:}
\expl{The more points, the better the estimates. Furthermore, an accurate
mean estimate requires less points than an accurate variance estimate. In
general, in any data-based pattern classification technique (as opposed to
knowledge-based techniques or expert systems), it is very important to have
enough training examples to estimate some accurate models of the data.}

\pagebreak
%%%%%%%%%
\subsection{Likelihood of a sample with respect to a Gaussian model}
\label{like}
%%%%%%%%%
\subsubsection*{Useful formulas and definitions\,:}
\begin{itemize}
\item[-] {\em Likelihood}\,: the likelihood of a sample point given a
Gaussian model (i.e. given a set of parameters $\Theta = (\mu,\Sigma)$) is
the value of the probability density function for that point. In the case
of Gaussian models, this amounts to compute the value of the pdf expression
given at the beginning of section~\ref{samples}.
\item[-] {\em Joint likelihood}\,: for a set of independent identically
distributed (i.i.d.) points, say $X = \{ x_1, x_2, \cdots, x_N \}$, the
joint (or total) likelihood is the product of the likelihood for each
point. For instance, in the Gaussian case\,:
\[ p(X|\Theta) = 
\prod_{i=1}^{N} p(x_i|\Theta) =
\prod_{i=1}^{N} p(x_i|\mu,\Sigma) =
\prod_{i=1}^{N} g_{(\mu,\Sigma)}(x_i)
\]
\end{itemize}

\subsubsection*{Experiment\,:}
Given the 4 Gaussian models\,:
\begin{center}
\begin{tabular}{ccc}
${\cal N}_1: \; \Theta_1 = \left(
\left[\begin{array}{c}730 \\ 1090\end{array}\right],
\left[\begin{array}{cc}8000 & 0 \\ 0 & 8000\end{array}\right]
\right)$ & \hspace{2cm} &
${\cal N}_2: \; \Theta_2 = \left(
\left[\begin{array}{c}730 \\ 1090\end{array}\right],
\left[\begin{array}{cc}8000 & 0 \\ 0 & 18500\end{array}\right]
\right)$ \\[2em]
${\cal N}_3: \; \Theta_3 = \left(
\left[\begin{array}{c}730 \\ 1090\end{array}\right],
\left[\begin{array}{cc}8000 & 8400 \\ 8400 & 18500\end{array}\right]
\right)$ & \hspace{2cm} &
${\cal N}_4: \; \Theta_4 = \left(
\left[\begin{array}{c}270 \\ 1690\end{array}\right],
\left[\begin{array}{cc}8000 & 8400 \\ 8400 & 18500\end{array}\right]
\right)$
\end{tabular}
\end{center}
\vspace{0.5em} compute the following {log-likelihoods} for the whole sample
$X_3$ (10000 points)\,:

\medskip
\centerline{$\log p(X_3|\Theta_1)$, $\log p(X_3|\Theta_2)$, $\log
p(X_3|\Theta_3)$ and $\log p(X_3|\Theta_4)$.}

\medskip
\noindent (First answer the following question and then look at the exemple
given on the next page.)

\vspace{-1ex}
\subsubsection*{Question\,:}
Why do we want to compute the {\em log-likelihood} rather than the simple
{\em likelihood} ?

\subsubsection*{Answer\,:}
\expl{
Computing the log-likelihood turns the product into a sum\,:
\[
p(X|\Theta) = \prod_{i=1}^{N} p(x_i|\Theta)
\;\;\; \Leftrightarrow \;\;\;
\log p(X|\Theta) = \sum_{i=1}^{N} \log p(x_i|\Theta)
\]
In the Gaussian case, it also avoids the computation of the exponential\,:
\begin{eqnarray}
p(x|\Theta) & = & \frac{1}{\sqrt{2\pi}^d \sqrt{\det\left(\Sigma\right)}}
\, e^{-\frac{1}{2} (x-\mu)^T \Sigma^{-1} (x-\mu)} \nonumber \\
\log p(x|\Theta) & = &
- \frac{d}{2} \log \left( 2\pi \right)
- \frac{1}{2} \log \left( \det\left(\Sigma\right) \right)
- \frac{1}{2} (x-\mu)^T \Sigma^{-1} (x-\mu) \nonumber
\end{eqnarray}
Furthermore, since $\log(x)$ is a monotonically growing function, the
log-likelihoods have the same relations of order as the likelihoods\,:
\[
p(x|\Theta_1) > p(x|\Theta_2) \;\; \Leftrightarrow \;\;
\log p(x|\Theta_1) > \log p(x|\Theta_2)
\]
so they can be used directly to classify samples.

\tab In a classification framework, the computation can be even more
simplified\,: the relations of order will remain valid if we drop the
division by 2 and the $d \log (2\pi)$ term (we have a right to drop these
terms because they are {\em independent of the classes}). If ${\cal
N}_1(\Theta_1)$ and ${\cal N}_2(\Theta_2)$ have the same variance, we can
also drop the $\log (\det(\Sigma))$ term (since in this case the variance
itself becomes independent of the classes).

\tab As a summary, log-likelihoods use simpler computation but are readily
usable for classification tasks.}

\pagebreak
\subsubsection*{Example\,:}
\mat{N = size(X3,1)}
\mat{mu\_1 = [730 1090]; sigma\_1 = [8000 0; 0 8000];}
\mat{logLike1 = 0;}
\mat{for i = 1:N;}
\com{logLike1 = logLike1 + (X3(i,:) - mu\_1)*inv(sigma\_1)*(X3(i,:) - mu\_1)';} \\
\com{end;} \\
\mat{logLike1 =  - 0.5 * ( logLike1 + N*log(det(sigma\_1)) + 2*N*log(2*pi) )}

\noindent Note\,: if you don't understand why the different models generate
a different log-likelihood value for the data, use the function
\com{gausview} to compare the relative positions of the models ${\cal
N}_1$, ${\cal N}_2$, ${\cal N}_3$ and ${\cal N}_4$ with respect to the data
set $X_3$, e.g.\,: \\
%
\mat{mu\_1 = [730 1090]; sigma\_1 = [8000 0; 0 8000];}
\mat{gausview(X3,mu\_1,sigma\_1,'Comparison of X3 and N1');}

\vspace{-\baselineskip}
\subsubsection*{Question\,:}
Of ${\cal N}_1$, ${\cal N}_2$, ${\cal N}_3$ and ${\cal N}_4$, which model
``explains'' best the data $X_3$ ?  Which model has the highest number of
parameters ?  Which model would you choose for a good compromise between
the number of parameters and the capacity to represent accurately the
data~?

\subsubsection*{Answer\,:}
\expl{The model ${\cal N}_3$ produces the highest likelihood for the data
set $X_3$. So we can say that data $X_3$ is more likely to have been
generated by model ${\cal N}_3$ than by the other models, or that ${\cal
N}_3$ explains best the data.

\tab On the other hand, model ${\cal N}_3$ has the highest number of
parameters (2 terms for the mean, 4 non null terms for the variance). This
may seem low in two dimensions, but the number of parameters grows
exponentially with the dimension of the data (this phenomenon is called the
{\em curse of dimensionality}). Also, the more parameters you have, the
more data you need to estimate (or {\em train}) them.

\tab In ``real world'' speech recognition applications, the dimensionality
of the speech features is typically of the order of 40 (1 energy
coefficient + 12 cepstrum coefficients + their first and second order
derivatives = a vector of 39 coefficients). Further processing is applied
to {\em orthogonalize} the data (e.g., cepstral coefficients can be
interpreted as orthogonalized spectra, and hence admit quasi-diagonal
covariance matrices).  Therefore, the compromise usually considered is to
use models with diagonal covariance matrices, such as the model ${\cal
N}_2$ in our example.}


%%%%%%%%%
%%%%%%%%%
\section{Statistical pattern recognition}
%%%%%%%%%
%%%%%%%%%

%%%%%%%%%
\subsection{A-priori class probabilities}
\label{sub:apriori}
%%%%%%%%%
\subsubsection*{Experiment\,:}
Load data from file ``vowels.mat''. This file contains a database of
simulated 2-dimensional speech features in the form of {\em artificial}
pairs of formant values (the first and the second spectral formants,
$[F_1,F_2]$). These artificial values represent the features that would be
extracted from several occurrences of vowels /a/, /e/, /i/, /o/ and
/y/\footnote{/y/ is the phonetic symbol for ``u'' like in the French word
``{\it tutu}''.}.  They are grouped in matrices of size $N\times2$, where
each of the $N$ lines is a training example and $2$ is the dimension of the
features (in our case, formant frequency pairs).

Supposing that the whole database covers adequately an imaginary language
made only of /a/'s, /e/'s, /i/'s, /o/'s and /y/'s, compute the probability
$P(q_k)$ of each class $q_k$, $k \in \{/a/,/e/,/i/,/o/,/y/\}$. What are the
most common and the least common phoneme in the language~?

\subsubsection*{Example\,:}
\mat{clear all; load vowels.mat; whos}
\mat{Na = size(a,1); Ne = size(e,1); Ni = size(i,1); No = size(o,1); Ny = size(y,1);}
\mat{N = Na + Ne + Ni + No + Ny;}
\mat{Pa = Na/N}
\mat{Pi = Ni/N}
etc.


\subsubsection*{Answer\,:}
\expl{The probability of using /a/ in this imaginary speech is 0.25.  It is
0.3 for /e/, 0.25 for /i/, 0.15 for /o/ and 0.05 for /y/. The most common
phoneme is therefore /e/, while the least common is /y/.}


%%%%%%%%%
\subsection{Gaussian modeling of classes}
\label{gaussmod}
%%%%%%%%%

\subsubsection*{Experiment\,:}
Plot each vowel's data as clouds of points in the 2D plane. Train the
Gaussian models corresponding to each class (use directly the \com{mean}
and \com{cov} commands). Plot their contours (use directly the function
\com{plotgaus({\em mu},{\em sigma},{\em color})} where \com{{\em color} =
[R,G,B]}).

\subsubsection*{Example\,:}
\mat{plotvow; \% Plot the clouds of simulated vowel features}
(Do not close the obtained figure, it will be used later on.) \\
Then compute and plot the Gaussian models\,: \\
\mat{mu\_a = mean(a);}
\mat{sigma\_a = cov(a);}
\mat{plotgaus(mu\_a,sigma\_a,[0 1 1]);}
\mat{mu\_e = mean(e);}
\mat{sigma\_e = cov(e);}
\mat{plotgaus(mu\_e,sigma\_e,[0 1 1]);}
etc.

\subsubsection*{Note your results below\,:}

\bigskip
\centerline{$\mu_{/a/} = $ \hfill $\Sigma_{/a/} =$ \hfill}

\vspace{1cm}
\centerline{$\mu_{/e/} = $ \hfill $\Sigma_{/e/} =$ \hfill}

\vspace{1cm}
\centerline{$\mu_{/i/} = $ \hfill $\Sigma_{/i/} =$ \hfill}

\vspace{1cm}
\centerline{$\mu_{/o/} = $ \hfill $\Sigma_{/o/} =$ \hfill}

\vspace{1cm}
\centerline{$\mu_{/y/} = $ \hfill $\Sigma_{/y/} =$ \hfill}

\vspace{1.5cm}

%%%%%%%%%
\subsection{Bayesian classification}
%%%%%%%%%

\subsubsection*{Useful formulas and definitions\,:}
\begin{itemize}
\item[-] {\em Bayes' decision rule}\,:
\[
	X \in q_k \;\; \mbox{if} \;\; P(q_k|X,\Theta) \geq P(q_j|X,\Theta), \; \forall j \neq k
\]
This formula means\,: given a set of classes $q_k$, characterized by a set
of known parameters $\Theta$, a set of one or more speech feature vectors
$X$ (also called {\em observations}) belongs to the class which has the
highest probability once we actually know (or ``see'', or ``measure'') the
sample $X$. $P(q_k|X,\Theta)$ is therefore called the {\em a posteriori
probability}, because it depends on having seen the observations, as
opposed to the {\em a priori} probability $P(q_k|\Theta)$ which does not
depend on any observation (but depends of course on knowing how to
characterize all the classes $q_k$, which means knowing the parameter set
$\Theta$).
\item[-] For some classification tasks (e.g. speech recognition), it is
more practical to resort to {\em Bayes' law}, which makes use of {\em
likelihoods}, rather than trying to estimate directly the posterior
probability. Bayes' law says\,:
\[
P(q_k|X,\Theta) = \frac{p(X|q_k,\Theta) P(q_k|\Theta)}{p(X|\Theta)}
\] 
where $q_k$ is a class, $X$ is a sample containing one or more feature
vectors and $\Theta$ is the parameter set of all the class models.
%
\item[-] The speech features are usually considered equi-probable. Hence,
it is considered that $P(q_k|X,\Theta)$ is proportional to $p(X|q_k,\Theta)
P(q_k|\Theta)$ for all the classes\,:
\[
\forall k, \;\; P(q_k|X,\Theta) \propto p(X|q_k,\Theta) P(q_k|\Theta)
\]
%
\item[-] Once again, it is more convenient to do the computation in the
$\log$ domain\,:
\[
\log P(q_k|X,\Theta) \simeq \log p(X|q_k,\Theta) + \log P(q_k|\Theta)
\]
\end{itemize}

\subsubsection*{Question\,:}
\begin{enumerate}
\item In our case (Gaussian models for phoneme classes), what is the
meaning of the $\Theta$ given in the above formulas ?
\item What is the expression of $p(X|q_k,\Theta)$, and of $\log
p(X|q_k,\Theta)$ ?
\item What is the definition of the probability $P(q_k|\Theta)$ ?
\end{enumerate}

\subsubsection*{Answer\,:}
\expl{
\begin{enumerate}
\item In our case, $\Theta$ represents the set of all the means $\mu_k$ and
variances $\Sigma_k$, $k \in \{/a/,/e/,/i/,/o/,/u/\}$.
\item The expression of $p(X|q_k,\Theta)$ and of $\log p(X|q_k,\Theta)$
correspond to the computation of the Gaussian pdf and its logarithm,
already expressed in section~\ref{like}.
\item The probability $P(q_k|\Theta)$ is the a-priori class probability for
the class $q_k$ (corresponding to the parameters $\Theta_k \in \Theta$). It
defines an absolute probability of occurrence for the class $q_k$. The
a-priori class probabilities for our artificial phoneme classes have been
computed in the section~\ref{sub:apriori}.
\end{enumerate}
}

\subsubsection*{Question\,:}
Now, we have modeled each vowel class with a Gaussian pdf (by computing
means and variances), we know the probability $P(q_k)$ of each class in the
imaginary language, and we assume that the speech {\em features} (as
opposed to speech {\em classes}) are equi-probable. What is the most
probable class $q_k$ for the speech feature points $x=(F_1,F_2)^T$ given in
the following table ? (Compute the posterior probabilities according to the
example given on the next page.)

\medskip
\noindent
\begin{center}
\renewcommand{\arraystretch}{1.5}
\begin{tabular}{|c|c|c|c|c|c|c|c|c|} \hline
 x & $F_1$ & $F_2$ &
\small $\log P(q_{/a/}|x)$ & \small $\log P(q_{/e/}|x)$ &
\small $\log P(q_{/i/}|x)$ & \small $\log P(q_{/o/}|x)$ &
\small $\log P(q_{/y/}|x)$ & \parbox[c][3em][c]{11ex}{Most prob. \\ class} \\ \hline
1. & 400 & 1800 & & & & & & \\ \hline
2. & 400 & 1000 & & & & & & \\ \hline
3. & 530 & 1000 & & & & & & \\ \hline
4. & 600 & 1300 & & & & & & \\ \hline
5. & 670 & 1300 & & & & & & \\ \hline
6. & 420 & 2500 & & & & & & \\ \hline
\end{tabular}
\end{center}

\subsubsection*{Example\,:}
Use function \com{gloglike({\em point},{\em mu},{\em sigma})} to compute
the likelihoods. Don't forget to add the log of the prior probability !
E.g., for point 1. and class /a/\,: \\
\com{>> gloglike([400,1800],mu\_a,sigma\_a) + log(Pa)}

\subsubsection*{Answer\,:}
\expl{1. /e/ \,\, 2. /y/ \,\, 3. /o/ \,\, 4. /y/ \,\, 5. /a/ \,\, 6. /i/}


\bigskip
%%%%%%%%%
\subsection{Discriminant surfaces}
\label{discr}
%%%%%%%%%

\subsubsection*{Useful formulas and definitions\,:}
\begin{itemize}
\item[-] {\em Discriminant function}\,: a set of functions $f_k(x)$ allows
to classify a sample $x$ into $k$ classes $q_k$ if\,:
\[
x \in q_k \;\; \Leftrightarrow \;\; f_k(x,\Theta_k) \geq f_l(x,\Theta_l),
\; \forall l \neq k
\]
In this case, the $k$ functions $f_k(x)$ are called discriminant functions.
\end{itemize}

\subsubsection*{Question\,:}
What is the link between discriminant functions and Bayesian classifiers~?

\subsubsection*{Answer\,:}
\expl{The a-posteriori probability $P(q_k|x)$ that a sample $x$ belongs to
class $q_k$ is itself a discriminant function\,:
\begin{eqnarray}
x \in q_k & \Leftrightarrow & P(q_k|x) \geq P(q_l|x) \; \forall l \neq k \nonumber \\
 & \Leftrightarrow & p(x|q_k) P(q_k) \geq p(x|q_l) P(q_l) \nonumber \\
 & \Leftrightarrow & \log p(x|q_k) + \log P(q_k) \geq \log p(x|q_l) + \log P(q_l) \nonumber
\end{eqnarray}
}

\subsubsection*{Experiment\,:}
%%%%%%%%%%%%
\begin{figure}
\centerline{\includegraphics[height=0.98\textheight]{iso.eps.gz}}
\caption{\label{iso}Iso-likelihood lines for the Gaussian pdfs ${\cal
N}\left(\mu_{/i/},\Sigma_{/i/}\right)$ and ${\cal N}\left(\mu_{/e/},\Sigma_{/e/}\right)$,
then ${\cal N}\left(\mu_{/i/},\Sigma_{/e/}\right)$ and ${\cal
N}\left(\mu_{/e/},\Sigma_{/e/}\right)$.}
\end{figure}
%%%%%%%%%%%%
The iso-likelihood lines for the Gaussian pdfs ${\cal
N}\left(\mu_{/i/},\Sigma_{/i/}\right)$ and ${\cal N}\left(\mu_{/e/},\Sigma_{/e/}\right)$,
which we used before to model the class /i/ and the class /e/, are plotted
on the next page (figure~\ref{iso}). On a second graph, the iso-likelihood
lines for ${\cal N}\left(\mu_{/i/},\Sigma_{/e/}\right)$ and ${\cal
N}\left(\mu_{/e/},\Sigma_{/e/}\right)$ (two pdfs with the same covariance matrix
$\Sigma_{/e/}$) are represented. On these figures, use a colored pen to join
the intersections of the level lines that correspond to equal likelihoods.

\subsubsection*{Question\,:}
What is the nature of the surface that separates class /i/ from class /e/
when the two models have {\em different} variances ? Can you explain the
origin of this form ?

What is the nature of the surface that separates class /i/ from class /e/
when the two models have the {\em same} variances ? Why is it different
from the previous discriminant surface ?

\subsubsection*{Answer\,:}
\expl{
In the case of different variances, the discriminant surface is a
parabola. As a matter of fact, in the Gaussian case\,:
\[
\log p(x|\Theta)
\simeq
\; - \frac{1}{2} (x-\mu)^T \Sigma^{-1} (x-\mu) \nonumber
\; - \frac{1}{2} \log \left( \det\left(\Sigma\right) \right)
\]
which is a quadratic form. Therefore, the equation\,:
\[
\log p(x|\Theta_1) = \log p(x|\Theta_2)
\]
which describes the discriminant surface is necessarily a second order
equation. Hence, its solution describes a parabolic discriminant surface.

\tab In the case where the covariance matrices are equal, the separation
between the class 1 and the class 2 does not depend upon the covariance
$\Sigma$ any more\,:
\hfill {\em (Continued on next page...)}
}

\expl{
{\bf Answer, continued\,:}
\begin{align*}
& & - \frac{1}{2} (x-\mu_1)^T \Sigma^{-1} (x-\mu_1)
\; - \frac{1}{2} \log \left( \det\left(\Sigma\right) \right)
\; & =
\; - \frac{1}{2} (x-\mu_2)^T \Sigma^{-1} (x-\mu_2)
\; - \frac{1}{2} \log \left( \det\left(\Sigma\right) \right)
\\
\\
\Leftrightarrow & &
- \frac{1}{2} (x-\mu_1)^T \Sigma^{-1} (x-\mu_1)
 & =
- \frac{1}{2} (x-\mu_2)^T \Sigma^{-1} (x-\mu_2)
\\
\\
\Leftrightarrow & &
- \frac{1}{2} x^T \Sigma^{-1} x
+ \mu_1^T \Sigma^{-1} x
- \frac{1}{2} \mu_1^T \Sigma^{-1} \mu_1
& =
- \frac{1}{2} x^T \Sigma^{-1} x
+ \mu_2^T \Sigma^{-1} x
- \frac{1}{2} \mu_2^T \Sigma^{-1} \mu_2
\end{align*}

\medskip
Hence, the equation of the surface that separates class $q_1$ from class
$q_2$ becomes\,:
\[
( \mu_1 - \mu_2 )^T \Sigma^{-1} x
- \frac{1}{2} ( \mu_1 - \mu_2 )^T \Sigma^{-1} ( \mu_1 - \mu_2 )
= 0
\]
which has a linear form. In the equal covariance case, using a Bayesian
classifier with Gaussian densities is therefore equivalent to using
discriminant functions of the form\,:
\[
f_k(x) = w_k^T x + w_{k0}
\]
where
\begin{eqnarray}
w_k^T  & = & \mu_k^T \Sigma^{-1}\nonumber \\
w_{k0} & = & - \frac{1}{2} \mu_k^T \Sigma^{-1} \mu_k \nonumber
\end{eqnarray}

\tab As a summary, we have shown that Bayesian classifiers with Gaussian
models separate the classes with combinations of parabolic surfaces. If the
covariance matrices of the models are equal, the parabolic separation
surfaces become simple hyper-planes.

}

\bigskip
%%%%%%%%%
%%%%%%%%%
\section{Unsupervised training}
\label{unsup}
%%%%%%%%%
%%%%%%%%%
In the previous section, we have computed the models for classes /a/, /e/,
/i/, /o/ and /y/ by knowing a-priori which training samples belongs to
which class (we were disposing of a {\em labeling} of the training
data). Hence, we have performed a {\em supervised training} of Gaussian
models.  Now, suppose that we only have unlabeled training data that we
want to separate in several classes (e.g., 5 classes) without knowing
a-priori which point belongs to which class. This is called {\em
unsupervised training}. Several algorithms are available for that purpose,
among which\,: the K-means, the Viterbi-EM and the EM
(Expectation-Maximization) algorithm.

%
%
\newcommand{\PBS}[1]{\let\temp=\\#1\let\\=\temp}
\newcommand{\RR}{\PBS\raggedright\hspace{0pt}}
\newcommand{\RL}{\PBS\raggedleft\hspace{0pt}}
\newcommand{\CC}{\PBS\centering\hspace{0pt}}
\begin{sidewaystable}[p]
\setlength{\arrayrulewidth}{1.2pt}
\renewcommand{\arraystretch}{2}
%\begin{center}
\hspace{-1.5em}
\begin{tabular}{|>{\RR}m{6em}|>{\RR}m{7.5em}|>{\CC}m{20em}|>{\RR}m{24.5em}|>{\CC}m{8em}|}
\hline
\centering\bf Algorithm &
\centering\bf Parameters &
\bf Membership measure &
\centering\bf Update method &
\bf Global criterion \\ \hline
%
Kmeans    &
$\bullet$ mean $\mu_k$ &
Euclidean distance \linebreak
\[d_k(x_n)= \sqrt{(x_n-\mu_k)^T(x_n-\mu_k)}\] \linebreak
(or the square of it)
&
Find the points closest to $q_k^{(old)}$, then\,:

\medskip
$\bullet$ $\mu_{k}^{(new)}$ = mean of the points closest to $q_k^{(old)}$ &
Least squares \\ \hline
% %
% Kmeans w/ \linebreak Mahalanobis distance &
% $\bullet$ mean $\mu_k$ \linebreak
% $\bullet$ variance $\Sigma_k$ &
% Mahalanobis distance, \linebreak (Weighted Euclidean distance)
% \[d_k(x_n)= \sqrt{(x_n-\mu_k)^T\Sigma_k^{-1}(x_n-\mu_k)}\]
% &
% Find the points closest to $q_k^{(old)}$, then\,:

% \medskip
% $\bullet$ $\mu_{k}^{(new)}$ = mean of the points closest to $q_k^{(old)}$

% \medskip
% $\bullet$ $\Sigma_{k}^{(new)}$ = variance of the points closest to $q_k^{(old)}$ &
% Minimal \linebreak spread of the data \linebreak (in a Mahalanobis distance sense) \\ \hline
%
Viterbi-EM &
$\bullet$ mean $\mu_k$ \linebreak
$\bullet$ variance $\Sigma_k$ \linebreak
$\bullet$ priors $P(q_k|\Theta)$
&
Posterior probability
\[
d_k(x_n) = P(q_k|x_n,\Theta)
\]
\footnotesize
\[
 \propto \frac{1}{\sqrt{2\pi}^d \sqrt{\det\left(\Sigma_k\right)}}
\, e^{-\frac{1}{2} (x_n-\mu_k)^T \Sigma_k^{-1} (x_n-\mu_k)} \cdot P(q_k|\Theta)
\]
\normalsize
&
Do Bayesian classification of each data point, then\,:

\medskip
$\bullet$ $\mu_{k}^{(new)}$ = mean of the points belonging to $q_k^{(old)}$

\medskip
$\bullet$ $\Sigma_{k}^{(new)}$ = variance of the points belonging to $q_k^{(old)}$

\medskip
$\bullet$ $P(q_k^{(new)}|\Theta^{(new)})$ = number of training points
belonging to $q_k^{(old)}$ / total number of training points
&
Maximum likelihood \\ \hline
%
EM &
$\bullet$ mean $\mu_k$ \linebreak
$\bullet$ variance $\Sigma_k$ \linebreak
$\bullet$ priors $P(q_k|\Theta)$ &
Posterior probability
\[
d_k(x_n) = P(q_k|x_n,\Theta)
\]
\footnotesize
\[
 \propto \frac{1}{\sqrt{2\pi}^d \sqrt{\det\left(\Sigma_k\right)}}
\, e^{-\frac{1}{2} (x_n-\mu_k)^T \Sigma_k^{-1} (x_n-\mu_k)} \cdot P(q_k|\Theta)
\]
\normalsize
&
Compute $P(q_k^{(old)}|x_n,\Theta^{(old)})$ (soft classification), then\,:

\medskip
$\bullet$ $\mu_{k}^{(new)} = \frac{\sum_{n=1}^{N} x_n P(q_k^{(old)}|x_n,\Theta^{(old)})}
								  {\sum_{n=1}^{N} P(q_k^{(old)}|x_n,\Theta^{(old)})} $

\medskip
$\bullet$ $\Sigma_{k}^{(new)} = \frac{\sum_{n=1}^{N} P(q_k^{(old)}|x_n,\Theta^{(old)})
								(x_n - \mu_k^{(new)})(x_n - \mu_k^{(new)})^T }
							    {\sum_{n=1}^{N} P(q_k^{(old)}|x_n,\Theta^{(old)})} $

\medskip
$\bullet$ $P(q_k^{(new)}|\Theta^{(new)}) = \frac{1}{N} \sum_{n=1}^{N} P(q_k^{(old)}|x_n,\Theta^{(old)}) $ &
Maximum likelihood \\ \hline
\end{tabular}
%\end{center}
\caption{\label{algos}Characteristics of some usual unsupervised clustering algorithms.}
\end{sidewaystable}
%
%
All these algorithms are characterized by the following components\,:
\begin{itemize}
\item a set of models $q_k$ (not necessarily Gaussian), defined by some
parameters $\Theta$ (means, variances, priors,...);
\item a measure of membership, telling to which extent a data point
``belongs'' to a model;
\item a ``recipe'' to update the model parameters in function of the
membership information.
\end{itemize}
The measure of membership usually takes the form of a measure of distance
or the form of a measure of probability. It replaces the missing labeling
information to permit the application of standard parameter estimation
techniques.  It also defines implicitly a global criterion of ``goodness
of fit'' of the models to the data, e.g.\,:
\begin{itemize}
\item in the case of a distance, the models that are globally closer from
the data characterize it better;
\item in the case of a probability measure, the models bringing a better
likelihood for the data explain it better.
\end{itemize}
Table~\ref{algos} summarizes the components of each of the algorithm that
will be studied in the following experiments. More detail will be given in
the corresponding subsections.

\pagebreak
%%%%%%%%%
\subsection{K-means algorithm}
%%%%%%%%%
\subsubsection*{Synopsis of the algorithm\,:}
\begin{itemize}
\item Start with $K$ initial prototypes $\mu_k$, $k=1,\cdots,K$.
\item {\bf Do}\,:
\begin{enumerate}
\item For each data-point $x_n$, $n=1,\cdots,N$, compute the squared Euclidean
distance from the $k^{th}$ prototype\,:
\begin{eqnarray}
\label{eq:dist}
d_k(x_n) & = & \, \left\| x_n - \mu_k \right\|^2 \nonumber \\
         & = & (x_n-\mu_k)^T(x_n-\mu_k)   \nonumber
\end{eqnarray}
\item Assign each data-point $x_n$ to its {\bf closest} prototype $\mu_k$,
i.e. assign $x_n$ to the class $q_k$ if\,:
\[
d_k(x_n) \, \leq \, d_l(x_n),
\;\; \forall l \neq k
\]
{\em Note}\,: using the square of the Euclidean distance for the
classification gives the same result as using the true Euclidean distance,
since the square root is a monotonically growing function. But the
computational load is obviously lighter when the square root is dropped.
\item Replace each prototype with the mean of the data-points assigned to
the corresponding class;
\item Go to 1.
\end{enumerate}
\item {\bf Until}\,: no further change occurs.
\end{itemize}
The global criterion defined in the present case is\,:
\[
	J = \sum_{k=1}^{K} \sum_{x_n \in q_k} d_k(x_n)
\]
and represents the total squared distance between the data and the models
they belong to. This criterion is locally minimized by the algorithm.

% Alternately, the Euclidean distance used in step 1. can be replaced by the
% Mahalanobis distance, belonging to the class of weighted Euclidean
% distances\,:
% \[
% 	d_k(x_n)= \sqrt{(x_n-\mu_k)^T\Sigma_k^{-1}(x_n-\mu_k)}
% \]
% where $\Sigma_k$ is the covariance of the points associated with the class
% $q_k$.

% After the convergence of the algorithm, the final clusters of points may be
% used to make Gaussian models since we dispose of means and variances.


\subsubsection*{Experiment\,:}
Use the K-means explorer utility\,:
\begin{verbatim}
 KMEANS K-means algorithm exploration tool

   Launch it with KMEANS(DATA,NCLUST) where DATA is the matrix
   of observations (one observation per row) and NCLUST is the
   desired number of clusters.

   The clusters are initialized with a heuristic that spreads
   them randomly around mean(DATA) with standard deviation
   sqrtm(cov(DATA)).

   If you want to set your own initial clusters, use
   KMEANS(DATA,MEANS) where MEANS is a cell array containing
   NCLUST initial mean vectors.

   Example: for two clusters
     means{1} = [1 2]; means{2} = [3 4];
     kmeans(data,means);

\end{verbatim}

\pagebreak
Launch it with the data sample \com{allvow}, which was part of file
\com{vowels.mat} and gathers all the simulated vowels data. Do several runs
with different cases of initialization of the algorithm\,:
\begin{enumerate}
\item 5 initial clusters determined according to the default heuristic;
\item some initial \com{MEANS} values equal to some data points;
\item some initial \com{MEANS} values equal to $\{\mu_{/a/}, \mu_{/e/},
\mu_{/i/}, \mu_{/o/}, \mu_{/y/}\}$.
\end{enumerate}
Iterate the algorithm until its convergence. Observe the evolution of the
cluster centers, of the data-points attribution chart and of the total
squared Euclidean distance. (It is possible to zoom these plots\,: left
click inside the axes to zoom $2\times$ centered on the point under the
mouse; right click to zoom out; click and drag to zoom into an area; double
click to reset the figure to the original). %Observe the mean and variance
% values found after the convergence of the algorithm.
Observe the mean values found after the convergence of the algorithm.


\subsubsection*{Example\,:}
\mat{kmeans(allvow,5);}
- or - \\
\mat{means =  \{ mu\_a, mu\_e, mu\_i, mu\_o, mu\_y \};}
\mat{kmeans(allvow,means);}
Enlarge the window, then push the buttons, zoom etc.
After the convergence, use\,:\\
\mat{for k=1:5, disp(kmeans\_result\_means\{k\}); end}
to see the resulting means.


\subsubsection*{Question\,:}
\begin{enumerate}
\item Does the final solution depend on the initialization of the
algorithm~?
\item Describe the evolution of the total squared Euclidean distance.
\item What is the nature of the discriminant surfaces corresponding to a
minimum Euclidean distance classification scheme~?
\item Is the algorithm suitable for fitting Gaussian clusters~?
\end{enumerate}


\subsubsection*{Answer\,:}
\expl{
\begin{enumerate}
\item The final solution depends upon the initialization of the
algorithm.
\item The total squared Euclidean distance decreases in a monotonic way.
\item The minimum-distance point attribution scheme is equivalent to linear
discriminant surfaces.
\item The final solution of the K-means algorithm is unable to lock
onto the original Gaussian phoneme classes.
\end{enumerate}
}


%%%%%%%%%
\subsection{Viterbi-EM algorithm for Gaussian clustering}
%%%%%%%%%
\subsubsection*{Synopsis of the algorithm\,:}
\begin{itemize}
\item Start from $K$ initial Gaussian models ${\cal N}(\mu_{k},\Sigma_{k}),
\; k=1\cdots K$, characterized by the set of parameters $\Theta$ (i.e. the
set of all means and variances $\mu_k$ and $\Sigma_k$, $k=1\cdots K$). Set
the initial prior probabilities $P(q_k)$ to $1/K$.
\item {\bf Do}\,:
\begin{enumerate}
\item Classify each data-point using Bayes' rule.

This step is equivalent to having a set $Q$ of boolean hidden variables
that give a labeling of the data by taking the value 1 (belongs) or 0 (does
not belong) for each class $q_k$ and each point $x_n$. The value of $Q$
that maximizes $p(X,Q|\Theta)$ precisely tells which is the most probable
model for each point of the whole set $X$ of training data.

Hence, each data point is assigned to its most probable cluster $q_k^{(old)}$.
\item Update the parameters\,:
\begin{itemize}
\item update the means\,:
	\[ \mu_{k}^{(new)} = \mbox{mean of the points belonging to } q_k^{(old)} \]
\item update the variances\,:
	\[ \Sigma_{k}^{(new)} = \mbox{variance of the points belonging to } q_k^{(old)} \]
\item update the priors\,:
	\[ P(q_k^{(new)}|\Theta^{(new)}) = \frac{\mbox{number of training points
	belonging to } q_k^{(old)} }{\mbox{total number of training points}}\]
\end{itemize}
\item Go to 1.
\end{enumerate}
\item {\bf Until}\,: no further change occurs.
\end{itemize}
The global criterion defined in the present case is\,:
\begin{eqnarray}
{\cal L}(\Theta) & = & \sum_{X} P(X|\Theta) \; = \; \sum_{Q} \sum_{X} p(X,Q|\Theta) \nonumber \\
      & = & \sum_{k=1}^{K} \sum_{x_n \in q_k} \log p(x_n|\Theta_k) \nonumber
\end{eqnarray}
and represents the joint likelihood of the data with respect to the models
they belong to. This criterion is locally maximized by the algorithm.


\subsubsection*{Experiment\,:}
Use the Viterbi-EM explorer utility\,:
\begin{verbatim}
 VITERB Viterbi version of the EM algorithm

   Launch it with VITERB(DATA,NCLUST) where DATA is the matrix
   of observations (one observation per row) and NCLUST is the
   desired number of clusters.

   The clusters are initialized with a heuristic that spreads
   them randomly around mean(DATA) with standard deviation
   sqrtm(cov(DATA)). Their initial covariance is set to cov(DATA).

   If you want to set your own initial clusters, use
   VITERB(DATA,MEANS,VARS) where MEANS and VARS are cell arrays
   containing respectively NCLUST initial mean vectors and NCLUST
   initial covariance matrices. In this case, the initial a-priori
   probabilities are set equal to 1/NCLUST.

   To set your own initial priors, use VITERB(DATA,MEANS,VARS,PRIORS)
   where PRIORS is a vector containing NCLUST a priori probabilities.

   Example: for two clusters
     means{1} = [1 2]; means{2} = [3 4];
     vars{1} = [2 0;0 2]; vars{2} = [1 0;0 1];
     viterb(data,means,vars);

\end{verbatim}
Launch it with the dataset \com{allvow}. Do several runs with different
cases of initialization of the algorithm\,:
\begin{enumerate}
\item 5 initial clusters determined according to the default heuristic;
\item some initial \com{MEANS} values equal to some data points, and some
random \com{VARS} values (try for instance \com{cov(allvow)} for all the
classes);
\item the initial \com{MEANS}, \com{VARS} and \com{PRIORS} values found by
the K-means algorithm.
\item some initial \com{MEANS} values equal to $\{\mu_{/a/}, \mu_{/e/},
\mu_{/i/}, \mu_{/o/}, \mu_{/y/}\}$, \com{VARS} values equal to \linebreak
$\{\Sigma_{/a/}, \Sigma_{/e/}, \Sigma_{/i/}, \Sigma_{/o/}, \Sigma_{/y/}\}$,
and \com{PRIORS} values equal to
$[P_{/a/},P_{/e/},P_{/i/},P_{/o/},P_{/y/}]$;
\item some initial \com{MEANS} and \com{VARS} values chosen by yourself.
\end{enumerate}
Iterate the algorithm until it converges. Observe the evolution of the
clusters, of the data points attribution chart and of the total likelihood
curve. Observe the mean, variance and priors values found after the
convergence of the algorithm. Compare them with the values computed in
section~\ref{gaussmod} (with supervised training).


\subsubsection*{Example\,:}
\mat{viterb(allvow,5);}
- or - \\
\mat{means =  \{ mu\_a, mu\_e, mu\_i, mu\_o, mu\_y \};}
\mat{vars = \{ sigma\_a, sigma\_e, sigma\_i, sigma\_o, sigma\_y \};}
\mat{viterb(allvow,means,vars);}
Enlarge the window, then push the buttons, zoom etc.
After convergence, use\,:\\
\mat{for k=1:5, disp(viterb\_result\_means\{k\}); end}
\mat{for k=1:5, disp(viterb\_result\_vars\{k\}); end}
\mat{for k=1:5, disp(viterb\_result\_priors(k)); end}
to see the resulting means, variances and priors.


\subsubsection*{Question\,:}
\begin{enumerate}
\item Does the final solution depend on the initialization of the
algorithm~?
\item Describe the evolution of the total likelihood. Is it monotonic~?
\item In terms of optimization of the likelihood, what does the final
solution correspond to~?
\item What is the nature of the discriminant surfaces corresponding to the
Gaussian classification~?
\item Is the algorithm suitable for fitting Gaussian clusters~?
\end{enumerate}


\subsubsection*{Answer\,:}
\expl{
\begin{enumerate}
\item The final solution strongly depends upon the initialization of
the algorithm.
\item The total likelihood increases monotonically, which means that
the models ``explain'' the training dataset better and better.
\item The final solution corresponds approximately to a local maximum
of likelihood in the sense of Bayesian classification with Gaussian
models.
\item As seen in section~\ref{discr}, the discriminant surfaces
corresponding to Gaussian clusters with unequal variances have the form of
(hyper-)parabolas.
\item The final solution of the Viterbi-EM algorithm is able to lock
approximately onto the simulated Gaussian phoneme classes if the number and
the placement of the initial clusters are correctly guessed (which is {\em
not an easy task}).
\end{enumerate}
}

\pagebreak
%%%%%%%%%
\subsection{EM algorithm for Gaussian clustering}
%%%%%%%%%
\subsubsection*{Synopsis of the algorithm\,:}
\begin{itemize}
\item Start from K initial Gaussian models ${\cal N}(\mu_{k},\Sigma_{k}),
\; k=1\cdots K$, with equal priors set to $P(q_k) = 1/K$.
\item {\bf Do}\,:
\begin{enumerate}
\item {\bf Estimation step}\,: compute the probability $P(q_k^{(old)}|x_n,\Theta^{(old)})$ for
each data point $x_n$ to belong to the class $q_k^{(old)}$\,:
%
\begin{eqnarray}
P(q_k^{(old)}|x_n,\Theta^{(old)}) & = & \frac{P(q_k^{(old)}|\Theta^{(old)})
										\cdot p(x_n|q_k^{(old)},\Theta^{(old)})}
										{p(x_n|\Theta^{(old)})} \nonumber \\
           & = & \frac{P(q_k^{(old)}|\Theta^{(old)}) \cdot p(x_n|\mu_k^{(old)},\Sigma_k^{(old)}) }
                      {\sum_j P(q_j^{(old)}|\Theta^{(old)}) \cdot p(x_n|\mu_j^{(old)},\Sigma_j^{(old)}) }
      \nonumber 
\end{eqnarray}

This step is equivalent to having a set $Q$ of continuous hidden variables,
taking values in the interval $[0,1]$, that give a labeling of the data by
telling to which extent a point $x_n$ belongs to the class $q_k$. This
represents a soft classification, since a point can belong, e.g., by 60\%
to class 1 and by 40\% to class 2 (think of Schr\"odinger's cat which is
60\% alive and 40\% dead as long as nobody opens the box or performs
Bayesian classification).
%
\item {\bf Maximization step}\,:
	\begin{itemize}
	\item update the means\,:
		\[\mu_{k}^{(new)} = \frac{\sum_{n=1}^{N} x_n P(q_k^{(old)}|x_n,\Theta^{(old)})}
								  {\sum_{n=1}^{N} P(q_k^{(old)}|x_n,\Theta^{(old)})} \]
	\item update the variances\,:
		\[\Sigma_{k}^{(new)} = \frac{\sum_{n=1}^{N} P(q_k^{(old)}|x_n,\Theta^{(old)})
								(x_n - \mu_k^{(new)})(x_n - \mu_k^{(new)})^T }
							    {\sum_{n=1}^{N} P(q_k^{(old)}|x_n,\Theta^{(old)})} \]
	\item update the priors\,:
		\[ P(q_k^{(new)}|\Theta^{(new)}) = \frac{1}{N} \sum_{n=1}^{N} P(q_k^{(old)}|x_n,\Theta^{(old)}) \]
	\end{itemize}
%
In the present case, all the data points participate to the update of all
the models, but their participation is weighted by the value of
$P(q_k^{(old)}|x_n,\Theta^{(old)})$.
\item Go to 1.
\end{enumerate}
\item {\bf Until}\,: the total likelihood increase for the training data
falls under some desired threshold.
\end{itemize}
The global criterion defined in the present case is\,:
\begin{eqnarray}
{\cal L}(\Theta) \; = \; \log p(X|\Theta)
				& = & \log \sum_Q p(X,Q|\Theta) \nonumber \\
				& = & \log \sum_Q P(Q|X,\Theta) p(X|\Theta) \;\;\;\; \mbox{(Bayes)} \nonumber \\
      			& = & \log \sum_{k=1}^{K} P(q_k|X,\Theta) p(X|\Theta) \nonumber
\end{eqnarray}
Applying Jensen's inequality $\left( \log \sum_j \lambda_j y_j \geq \sum_j
\lambda_j \log y_j \mbox{ if } \sum_j \lambda_j = 1 \right)$, we
obtain\,:
\begin{eqnarray}
{\cal L}(\Theta) & \approx & \sum_{k=1}^{K} P(q_k|X,\Theta) \log p(X|\Theta) \nonumber \\
      & = & \sum_{k=1}^{K} \sum_{n=1}^{N} P(q_k|x_n,\Theta) \log p(x_n|\Theta) \nonumber
\end{eqnarray}
Hence, the final $J$ represents a lower boundary for the joint likelihood of
all the data with respect to all the models. This criterion is locally
maximized by the algorithm.


\subsubsection*{Experiment\,:}
Use the EM explorer utility\,:
\begin{verbatim}
 EMALGO EM algorithm explorer

   Launch it with EMALGO(DATA,NCLUST) where DATA is the matrix
   of observations (one observation per row) and NCLUST is the
   desired number of clusters.

   The clusters are initialized with a heuristic that spreads
   them randomly around mean(DATA) with standard deviation
   sqrtm(cov(DATA)*10). Their initial covariance is set to cov(DATA).

   If you want to set your own initial clusters, use
   EMALGO(DATA,MEANS,VARS) where MEANS and VARS are cell arrays
   containing respectively NCLUST initial mean vectors and NCLUST
   initial covariance matrices. In this case, the initial a-priori
   probabilities are set equal to 1/NCLUST.

   To set your own initial priors, use VITERB(DATA,MEANS,VARS,PRIORS)
   where PRIORS is a vector containing NCLUST a priori probabilities.

   Example: for two clusters
     means{1} = [1 2]; means{2} = [3 4];
     vars{1} = [2 0;0 2]; vars{2} = [1 0;0 1];
     emalgo(data,means,vars);

\end{verbatim}
Launch it with again the same dataset \com{allvow}. Do several runs with
different cases of initialization of the algorithm\,:
\begin{enumerate}
\item 5 clusters determined according to the default heuristic;
\item some initial \com{MEANS} values equal to some data points, and some
random \com{VARS} values (e.g. \com{cov(allvow)} for all the classes);
\item the initial \com{MEANS} and \com{VARS} values found by the K-means algorithm.
\item some initial \com{MEANS} values equal to $\{\mu_{/a/}, \mu_{/e/},
\mu_{/i/}, \mu_{/o/}, \mu_{/y/}\}$, \com{VARS} values equal to \linebreak
$\{\Sigma_{/a/}, \Sigma_{/e/}, \Sigma_{/i/}, \Sigma_{/o/}, \Sigma_{/y/}\}$,
and \com{PRIORS} values equal to
$[P_{/a/},P_{/e/},P_{/i/},P_{/o/},P_{/y/}]$;
\item some initial \com{MEANS} and \com{VARS} values chosen by yourself.
\end{enumerate}
(If you have time, also increase the number of clusters and play again with
the algorithm.)

Iterate the algorithm until the total likelihood reaches an asymptotic
convergence. Observe the evolution of the clusters and of the total
likelihood curve. (In the EM case, the data points attribution chart is not
given because each data point participates to the update of each cluster.)
Observe the mean, variance and prior values found after the convergence of
the algorithm. Compare them with the values found in
section~\ref{gaussmod}.


\subsubsection*{Example\,:}
\mat{emalgo(allvow,5);}
- or - \\
\mat{means =  \{ mu\_a, mu\_e, mu\_i, mu\_o, mu\_y \};}
\mat{vars = \{ sigma\_a, sigma\_e, sigma\_i, sigma\_o, sigma\_y \};}
\mat{emalgo(allvow,means,vars);}
Enlarge the window, then push the buttons, zoom etc.
After convergence, use\,:\\
\mat{for k=1:5, disp(emalgo\_result\_means\{k\}); end}
\mat{for k=1:5, disp(emalgo\_result\_vars\{k\}); end}
\mat{for k=1:5, disp(emalgo\_result\_priors(k)); end}
to see the resulting means, variances and priors.


\subsubsection*{Question\,:}
\begin{enumerate}
\item Does the final solution depend on the initialization of the
algorithm~?
\item Describe the evolution of the total likelihood. Is it monotonic~?
\item In terms of optimization of the likelihood, what does the final
solution correspond to~?
\item Is the algorithm suitable for fitting Gaussian clusters~?
\end{enumerate}


\subsubsection*{Answer\,:}
\expl{
\begin{enumerate}
\item Here again, the final solution depends upon the initialization
of the algorithm.
\item Again, the total likelihood always increases monotonically.
\item The final solution corresponds to a local maximum of likelihood
in the sense of Bayesian classification with Gaussian models.
\item The final solution of the EM algorithm is able to lock
perfectly onto the Gaussian phoneme clusters, but only if the number and
the placement of the initial clusters are correctly guessed (which, once
again, is {\em not an easy task}).

\tab In practice, the initial clusters are usually set to the K-means
solution (which, as you have seen, may not be an optimal
guess). Alternately, they can be set to some ``well chosen'' data points.
\end{enumerate}
}

\vfill
%%%%%%%%%
%%%%%%%%%
\section*{After the lab...}
%%%%%%%%%
%%%%%%%%%
This lab manual can be kept as additional course material. If you want to
browse the experiments again, you can use the script\,:

\noindent \com{>> lab1demo}

\noindent
which will automatically redo all the computation and plots for you (except
those of section~\ref{unsup}).


\end{document}
